\documentclass{article}

\usepackage{amsmath}
%\usepackage{amsfonts}
\usepackage{amsthm}
%\usepackage{amssymb}
%\usepackage{mathrsfs}
%\usepackage{fullpage}
%\usepackage{mathptmx}
%\usepackage[varg]{txfonts}
\usepackage{color}
\usepackage[charter]{mathdesign}
\usepackage[pdftex]{graphicx}
%\usepackage{float}
%\usepackage{hyperref}
%\usepackage[modulo, displaymath, mathlines]{lineno}
%\usepackage{setspace}
%\usepackage[titletoc,toc,title]{appendix}
\usepackage{natbib}

%\linenumbers
%\doublespacing

\theoremstyle{definition}
\newtheorem*{defn}{Definition}
\newtheorem*{exm}{Example}

\theoremstyle{plain}
\newtheorem*{thm}{Theorem}
\newtheorem*{lem}{Lemma}
\newtheorem*{prop}{Proposition}
\newtheorem*{cor}{Corollary}

\newcommand{\argmin}{\text{argmin}}
\newcommand{\ud}{\hspace{2pt}\mathrm{d}}
\newcommand{\bs}{\boldsymbol}
\newcommand{\PP}{\mathsf{P}}
\let\divsymb=\div % rename builtin command \div to \divsymb
\renewcommand{\div}[1]{\operatorname{div} #1} % for divergence
\newcommand{\Id}[1]{\operatorname{Id} #1}

\title{Tensor product elements for the Stokes equations}
\author{}
\date{}

\begin{document}

\maketitle

In this paper we'll examine finite element discretizations of the Stokes equations on \emph{extruded meshes}.
We're especially interested in domains with a high aspect ratio, or \emph{thin-film} flows.
For example, terrestrial glacier flows have horizontal length scales on the order of hundreds of kilometers or more, and vertical length scales on the order of 1 km.
These kinds of flows are typically very smooth in the vertical dimension and can often be transformed into terrain-following coordinates.
For velocity fields that are very smooth in the vertical, it may be more computationally efficient to discretize using high-degree vertical basis functions in one vertical layer than it is to use many vertical layers with low-degree basis functions.
Finally, the use of high-degree basis functions in the vertical suggests using a p-multigrid scheme.

To our knowledge, the LBB-stability of tensor product elements with high-degree polynomial vertical basis functions has not been established in the computational fluid dynamics literature.
\citet{canuto1984combined} established the LBB-stability of a basis using the tensor product of the MINI element in the horizontal direction and a Fourier basis with an equal number of modes for the velocity and pressure in the vertical direction.
\citet{nakahashi1989finite} studied prismatic elements for the Navier-Stokes equations but did not address LBB stability.

We propose to use the velocity and pressure spaces
\begin{align}
    V_h & = \{\text{CG}_2 \otimes \text{GLL}_{k + 1}\}^3 \\
    Q_h & = \text{CG}_1 \otimes \text{GL}_k.
\end{align}
CG denotes the usual continuous Lagrange basis, while GLL and GL are respectively the continuous Gauss-Lobatto-Legendre and discontinuous Gauss-Lobatto bases on intervals.
Alternatively, one could use the MINI or other stable element in the horizontal.
In the following we'll need to use decompositions of the velocity space into horizontal and vertical components.
For our proofs to go, we have to assume that the velocity space $V$ has the structure
\begin{equation}
    V = W^3 \otimes V^z
\end{equation}
where $W$ is some scalar space defined on the footprint domain $\Omega$ and $V^z$ is a scalar space defined on $[0, 1]$.
We likewise have to assume that the discrete spaces satisfy a similar relation:
\begin{equation}
    V_h = W_h^3 \otimes V_h^z.
\end{equation}
This assumption holds for certain discretizations but not others.
For example, the Taylor-Hood and MINI spaces can be written as the Cartesian product of scalar spaces, while the Raviart-Thomas, Brezzi-Douglas-Marini, and Crouzeix-Raviart spaces cannot.
The reason that we make this assumption is that we will need to express the velocity space as a Cartesian product of a horizontal and a vertical part:
\begin{align}
    V & \cong \underbrace{(V^x \otimes V^z)}_{\text{horizontal}} \times \underbrace{(W \otimes V^z)}_{\text{vertical}} \\
    & \cong (W^2\otimes V^z) \times (W \otimes V^z)
    \label{eq:space-decomposition}
\end{align}
Let $\Psi_x : V \to V^x\otimes V^z$ and $\Psi_z : V \to W \otimes V^z$ be the operators that project onto the horizontal and vertical components of the velocity field respectively.
Note that $\Psi_x$ and $\Psi_z$ are mutually orthogonal projection operators:
\begin{align}
    \Psi_x\cdot\Psi_z^* = \Psi_z\cdot\Psi_x^* & = 0, \label{eq:orthogonal-projection} \\
    \Psi_x^*\cdot\Psi_x + \Psi_z^*\cdot\Psi_z & = \Id.
\end{align}
Finally, for the pressure space, we only need that $Q_h = Q_h^x\otimes Q_h^z$.

The Stokes equations can be written as a constrained optimization problem.
The velocity and pressure that solve the Stokes equations are a critical point of the Lagrangian
\begin{equation}
    L(u, p) = \frac{1}{2}\langle Au, u\rangle - \langle f, u\rangle - \langle Bu - g, p\rangle.
\end{equation}
In our case, $B$ is the 3D divergence operator and the pressure $p$ acts as a Lagrange multiplier to enforce the constraint that the velocity field is divergence-free.
The key obstruction to LBB-stability for the Stokes equations is coercivity for the constraint operator $B$.
We need that
\begin{equation}
    \inf_{q\in Q_h}\sup_{v\in V_h}\frac{\langle Bv, q\rangle}{\|v\|\cdot\|q\|} \ge \beta
    \label{eq:inf-sup}
\end{equation}
for some $\beta > 0$ independent of $h$, but not every choice of velocity-pressure element pairs satisfies this condition.
(This is to be contrasted to other problems with a similar form, such as mixed Poisson, where the obstruction is instead a failure of coercivity of the $A$ operator.)
To prove the LBB-stability of tensor product elements for the Stokes equations, we will assume that we have pairs of discrete spaces $(V_h^x, Q_h^x) \subset (V^x, Q^x)$ and $(V_h^z, Q_h^z) \subset (V^z, Q^z)$ for which the inf-sup condition \eqref{eq:inf-sup} holds respectively for the horizontal and vertical divergence operators $\div_x : V^x \to Q^x$, $\div_z : V^z \to Q^z$.
(The 1D divergence operator is trivially just differentiation in the $z$-direction.)
To use the space decomposition in equation \eqref{eq:space-decomposition}, we'll express the divergence operator in terms of how it acts on the horizontal and vertical spaces.
Let $\Id_{V^x}$, $\Id_{V^z}$, $\Id_W$ the identity operators on $V^x$, $V^z$, and $W$ respectively.
Then we can write the full 3D divergence operator as a sum of tensor products of operators:
\begin{equation}
    \div = (\div_x \otimes \Id_{V^z})\cdot\Psi_x + (\Id_W \otimes \div_z)\cdot\Psi_z
    \label{eq:divergence}
\end{equation}
This is the key identity that will allow the rest of the proofs to work in the following.

A common strategy for proving the stability of a discretization for a mixed problem is to show that a \emph{$B$-compatible interpolation operator} exists.
A $B$-compatible interpolation operator is a map $\Pi_h : V \to V_h$ such that, for all $q$ in $Q_h$,
\begin{equation}
    \langle B(v - \Pi_hv), q\rangle = 0
\end{equation}
and such that $\|\Pi_hv\| \le C\|v\|$ where the constant $C$ is independent of $h$.
First, suppose that there are such operators $\Pi_x$, $\Pi_z$ for $\div_x$, $\div_z$, and let $\Pi_w$ be an arbitrary bounded interpolation operator into the scalar space $W_h$.
We claim that the compatible interpolation operator for the full 3D divergence is
\begin{equation}
    \Pi_h = \Psi_x^*\cdot\Pi_x\otimes\Pi_z\cdot\Psi_x + \Psi_z^*\cdot\Pi_w\otimes\Pi_z\cdot\Psi_z.
    \label{eq:pi-h}
\end{equation}
Before proceeding, we note to useful facts.
First, the norm of a tensor product of operators is the product of the norms:
\begin{equation}
    \|T_1 \otimes T_2\| = \|T_1\|\cdot\|T_2\|.
\end{equation}
This together with the triangle inequality implies that $\Pi_h$ is bounded with norm independent of $h$ if $\Pi_x$, $\Pi_z$, and $\Pi_w$ are.
Second, operator product distributes over tensor product:
\begin{equation}
    (S_1 \otimes S_2) \cdot (T_1 \otimes T_2) = (S_1 \cdot T_1) \otimes (S_2 \cdot T_2).
    \label{eq:operator-tensor-product}
\end{equation}
To show that $\Pi_h$ is compatible for the 3D divergence operator, we'll first look at how the interpolation error decomposes in the horizontal and vertical directions respectively using equations \eqref{eq:orthogonal-projection}:
\begin{align}
    v - \Pi_h v & = (\Psi_x^*\cdot\Psi_x + \Psi_z^*\cdot\Psi_z - \Pi_h) v \\
    & = \Psi_x^*\cdot(\Id_{V^x}\otimes\Id_{V^z} - \Pi_x\otimes\Pi_z)\cdot\Psi_xv \nonumber \\
    & \qquad + \Psi_z^*\cdot(\Id_W\otimes\Id_{V^z} - \Pi_w\otimes\Pi_z)\cdot\Psi_zv
\end{align}
Now if we take the divergence of the expression above using the decomposition of equation \eqref{eq:divergence} we'll find that several cross-terms are zero because of equation \eqref{eq:orthogonal-projection}:
\begin{align}
    \div(v - \Pi_h v) & = \div_x\otimes \Id_{V^z}\cdot(\Id_{V^x}\otimes\Id_{V^z} - \Pi_x\otimes\Pi_z)\cdot\Psi_xv \nonumber \\
    & \qquad + \Id_W\otimes\div_z\cdot(\Id_W\otimes\Id_{V^z} - \Pi_w\otimes\Pi_z)\cdot\Psi_zv
\end{align}
\textcolor{red}{Then a miracle occurs.
I think we need to add zero in a clever way so that we can factor everything using equation \eqref{eq:operator-tensor-product}.
Then we'll use the compatibility of the $x$ and $z$ operators.}


\pagebreak

\bibliographystyle{plainnat}
\bibliography{tensor-product-stokes.bib}

\end{document}
